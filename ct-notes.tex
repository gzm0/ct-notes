\documentclass[fleqn]{scrartcl}

%% Handy math stuff
\usepackage{amsmath}
\usepackage{amsthm}
\usepackage{amsfonts}

%% Control margins in itemize
\usepackage{enumitem}

%% Commutative diagram tikz package
\usepackage{tikz-cd}

%% Category environment
\newenvironment{category}{\begin{itemize}[leftmargin=.9in]}{\end{itemize}}
\newcommand{\catobj}[1]{\item[\textbf{Object}] #1}
\newcommand{\catarr}[1]{\item[\textbf{Arrows}] #1}
\newcommand{\catcomp}[1]{\item[\textbf{Comp.}] #1}
\newcommand{\catid}[1]{\item[\textbf{Ident.}] #1}

%% Commonly used math symbols
% An abstract category
\newcommand{\cat}[1]{\mathcal{#1}}
% A concrete category
\newcommand{\ccat}[1]{\mathbf{#1}}
% Domain of an arrow
\newcommand{\dom}{\mathrm{dom}}
% Codomain of an arrow
\newcommand{\cod}{\mathrm{cod}}

% Therom styles
\theoremstyle{definition} \newtheorem{definition}{Definition}
\theoremstyle{remark} \newtheorem{example}{Example}

\begin{document}

\title{Introduction to Category Theory}
\date{}
\author{LAMP Team}

\maketitle

\section{Motivation / Goals}

\begin{itemize}
\item Not to teach Haskell / Scalaz
\item Not to develop deep theorems
\item Get some basic notions (compare Set theory in theoretical
  computer science)
\item Hopefully useful for reading papers
\end{itemize}

Category theory is about relation between different objects. Rather
than having elements, we think of functions or behavior. You could
call it the mathematical theory of abstraction. Category theory
unifies existing concepts and hopefully makes it applicable to other
things.

\section{Categories}

\begin{definition}[Category]
  A category $\cat{C}$ is
  \begin{category}
    \catobj{a collection $\cat{C}_0$ of objects}
    \catarr{a collection $\cat{C}_1$ of arrows (or morphisms,
      homomorphisms)}
    \catcomp{a composition operation $\circ$}
    \catid{an identity arrow $id_A: A \to A \in \cat{C}_1$ for every $A
      \in \cat{C}_0$}
  \end{category}
  Such that for each arrow $f \in \cat{C}_1$ there are objects
  $\dom(f),\ \cod(f) \in \cat{C}_0$. In other words, there are maps
  (total functions) $\dom,\ \cod : \cat{C}_1 \to  \cat{C}_0$. If $A =
  \dom(f),\ B = \cod(f)$ we write $f : A \to B$.
  
  Further, for every pair $f : A \to B$ and $g : B \to C$ of arrows,
  there is a  ``composite'' arrow
  \[ (g \circ f) : A \to C \in \cat{C}_1 \]
  Composition ($\circ$) is associative:
  \[ (g \circ f) \circ h = g \circ (f \circ h) \qquad (\forall g,f,h
  \in \cat{C}_1) \]
  
  For every object $A \in \cat{C}_0$ there is an arrow $id_A: A \to A$
  such that for any $f : A \to B$:
  \[ id_B \circ f = f \quad\mathrm{and}\quad f \circ id_A = f \]
\end{definition}

\begin{example}
  $\ccat{Set}$ is the category of sets and total functions.
  \begin{category}
    \catobj{$\ccat{Set}_0$ is the class of all sets}
    \catarr{$\ccat{Set}_1$ is the class of total functions on sets}
    \catcomp{$\circ$ is the (set) function composition}
    \catid{identity (set) function}
  \end{category}
  Given a total function $f$, we have:
  \[ \dom(f) = \dom(f) \]
  (the domain in the ``category world'' is the same as the domain in
  the ``function world'').
\end{example}

\begin{example}
  Category $\ccat{1}$
  \begin{category}
    \catobj{$\ccat{1}_0 = \{ \star \}$}
    \catarr{$\ccat{1}_1 = \{ id_\star \}$}
    \catcomp{$id_\star \circ id_\star = id_\star$}
    \catid{$id_\star$}
  \end{category}
\end{example}

\begin{example}
  Category $\ccat{2}_\to$
  \begin{category}
    \catobj{$\ccat{2}_0 = \{ \star_1, \star_2 \}$}
    \catarr{$\ccat{2}_1 = \{ id_1, id_2, f : \star_1 \to \star_2 \}$}
    \catcomp{follows from identity axiom}
    \catid{$\{ id_1, id_2 \}$}
  \end{category}
\end{example}

\section{Diagrams}
We can represent categories (or properties thereof) by diagrams. For
example:

\begin{tikzcd}
A \rar{f} \dar{g'} & B \dar{g} \\
C \rar{f'}         & D
\end{tikzcd}\\
which states that $g \circ f = f' \circ g'$.

\begin{example}
  $\ccat{Grph}$ is the category of multi-digraphs\footnote{Graphs with
    multiple, directed edges} and graph homomorphisms.

  \begin{category}
    \catobj{A graph $G$ is a pair of sets $G_0, G_1$ of nodes and arrows with a
      pair of maps $s_G, t_G : G_1 \to G_0$.}
    \catarr{A graph morphism $f : G \to H$ between graphs $G$ and $H$ is a
      pair
      \[ f_0 : G_0 \to H_0, \quad f_1 : G_1 \to H_1 \]
    of (set) functions, such that}

    \begin{tikzcd}
      G_1 \rar{f_1} \dar{s_G} & H_1 \dar{s_H} & G_1 \rar{f_1} \dar{t_G} & H_1 \dar{t_H} \\
      G_0 \rar{f_0}           & H_0           & G_0 \rar{f_0}           & H_0
    \end{tikzcd}

    \catcomp{Given two graph morphisms:
      \[ f : G \to H = (f_0, f_1) \]
      \[ g : H \to K = (g_0, f_1) \]
      we define their composition:
      \[ (g \circ f) = (g_0 \circ f_0 : G_0 \to K_0,\ g_1 \circ f_1 : G_1
      \to K_1) \]
    }
    \catid{The identity graph morphism $id_G : G \to G$ is defined as
      the pair $(id_0, id_1)$, where $id_0$ is the identity function
      over nodes and $id_1$ is the identity function over arrows}
  \end{category}

  We have to show that the composition is still a graph morphism and
  that it is associative. We have:

  \begin{tikzcd}
    G_1 \rar{f_1} \dar{s_G} & H_1 \dar{s_H} & H_1 \rar{g_1} \dar{s_H} & K_1 \dar{s_K} \\
    G_0 \rar{f_0}           & H_0           & H_0 \rar{g_0}           & K_0
  \end{tikzcd}

  By pasting the two together, we can show that the condition on source
  maps holds.

  \begin{tikzcd}
    G_1 \rar{f_1} \dar{s_G} & H_1 \dar{s_H} \rar{g_1} & K_1 \dar{s_K} \\
    G_0 \rar{f_0}           & H_0           \rar{g_0}  & K_0
  \end{tikzcd}

  Equivalent proof for target maps. This proves that a composed graph
  morphism is a graph morphism.

  We show $id_H \circ f = f$ using graphs:

  \begin{tikzcd}
    G_1 \rar{f_1} \dar{s_G} & H_1 \dar{s_H} \rar[equal]{id_{H_1}} & H_1 \dar{s_H} \\
    G_0 \rar{f_0}           & H_0           \rar[equal]{id_{H_1}} & H_0
  \end{tikzcd}

  We omit proofs of associativity.
\end{example}

\section{Products}

\begin{example}
  Given sets $S, T$ the cartesian product $S \times T$ is the set
  \[ S \times T = \{ (s, t) | s \in S, t \in T \} \]
\end{example}

We generalize this notion to products in categories.

\begin{definition}[Product]
  Given two objects $A, B$ in a category $\cat{C}$, \textbf{a} product is:
  \begin{itemize}
  \item an object $C \in \cat{C}_0$ (often denoted $A \times B$)
  \item an arrow $proj_1 : C \to A \in \cat{C}_1$ (sometimes denoted $\pi_1$)
  \item an arrow $proj_2 : C \to B \in \cat{C}_1$ (sometimes denoted $\pi_2$)
  \end{itemize}
  such that for any 
  \[ D \in \cat{C}_0,\ q_1 : D \to A \in \cat{C}_1,\ q_2 : D \to B \in
  \cat{C}_1 \]
  there is a \emph{unique} $q : D \to C$ such that
  \[ \pi_1 \circ q = p_1 \ \mathrm{and} \ \pi_2 \circ q = p_2 \]
  We can express this with the following commuting diagram:

  \begin{tikzcd}
    \phantom{A} & D \ar{dl}{q_1}\dar{q}\ar{dr}{q_2} & \\
    A & C \lar{\pi_1}\rar{\pi_2}          & B
  \end{tikzcd}
\end{definition}

\begin{example}
Is the cartesian product is a product in the category of sets? Yes!\\
Given two sets $S,\ T$, the cartesian product $S \times T$ is a
product:

\begin{itemize}
\item $S \times T$ is a set
\item $\pi_1 : S \times T \to S \quad \pi_1((s, t)) = s$
\item $\pi_2 : S \times T \to T \quad \pi_1((s, t)) = t$
\end{itemize}

For example:
\begin{align*}
A &= Int\\
B &= String\\
A \times B &= Int \times String\\
D &= Boolean\\
q_1 &= Boolean \to Int \textrm{ (C-encoding of boolean)}\\
q_2 &= Boolean \to String \textrm{ (toString)}\\
q &= Boolean \to Int \times String
\end{align*}
\end{example}

\section{Isomorphisms}
\begin{definition}[Isomorphism]
An iso in a category $\cat{C}$,
\[ f: A \to B \in \cat{C}_1 \]
is an arrow such that there is an arrow
\[ f^{-1} : B \to A \]
such that 
\[ f \circ f^{-1} = id_B \textrm{ and } f^{-1} \circ f = id_A. \]
If there is an iso $f : A \to B$ we say $A$ and $B$ are isomorphic and
we write $A \cong B$ or $f : A \stackrel{\sim}{\to} B$.
\end{definition}

\end{document}

%%% Local Variables: 
%%% mode: latex
%%% TeX-master: t
%%% End: 
