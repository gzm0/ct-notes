\documentclass{scrartcl}

%% Handy math stuff
\usepackage{sty/mathcommon}

%% Control margins in itemize
\usepackage{enumitem}

%% Commutative diagram tikz package
\usepackage{tikz-cd}

%% Category environment
\newenvironment{category}{\begin{itemize}[leftmargin=.9in]}{\end{itemize}}
\newcommand{\catobj}[1]{\item[\textbf{Objects}] #1}
\newcommand{\catarr}[1]{\item[\textbf{Arrows}] #1}
\newcommand{\catcomp}[1]{\item[\textbf{Comp.}] #1}
\newcommand{\catid}[1]{\item[\textbf{Identity}] #1}

% Manually insert an equation number (e.g. for numbering diagrams) and
% label it.
\newcommand{\eqnlabel}[1]{%
  \refstepcounter{equation}\textup{(\theequation)}\label{#1}}

% Environment for diagrams (centers the diagrams contained withing and
% removes some vertical space before and after).
\newenvironment{diags}[1][0pt]{\begin{center}\vspace{#1}%
    \def\diagsspaceafter{#1}}{%
    \vspace{\diagsspaceafter}\end{center}}


\begin{document}

\title{Introduction to Category Theory}
\date{}
\author{LAMP Team}

\maketitle

\section{Motivation / Goals}

\begin{itemize}
\item Not to teach Haskell / Scalaz
\item Not to develop deep theorems
\item Get some basic notions (compare Set theory in theoretical
  computer science)
\item Hopefully useful for reading papers
\end{itemize}

Category theory is about relation between different objects. Rather
than having elements, we think of functions or behavior. You could
call it the mathematical theory of abstraction. Category theory
unifies existing concepts and hopefully makes it applicable to other
things.

\section{Categories}

\begin{definition}[Category]
  A \emph{category} $\cat{C}$ is
  \begin{category}
    \catobj{a collection $\obs{\cat{C}}$ of \emph{objects},}
    \catarr{a collection $\arrs{\cat{C}}$ of \emph{arrows} (or
      morphisms, homomorphisms),}
    \catcomp{a binary composition operation $\circ$ on arrows,}
    \catid{an identity arrow $\id_A \in \arrs{\cat{C}}$ for every
      object $A \in \obs{\cat{C}}$,}
  \end{category}
  such that for each arrow $f \in \arrs{\cat{C}}$ there are objects
  $\dom(f), \cod(f) \in \obs{\cat{C}}$ called the \emph{domain} and
  \emph{codomain} of $f$, respectively.  In other words, there are
  maps (total functions) $\dom, \cod \colon \arrs{\cat{C}} \to
  \obs{\cat{C}}$.  If $A = \dom(f),\ B = \cod(f)$, we write $f \colon
  A \to B$.

  Further, for every pair $f \colon A \to B$ and $g \colon B \to C$ of
  arrows, there is a \emph{composite} arrow
  \[ (g \circ f) \colon A \to C \in \arrs{\cat{C}}. \]
  Composition ($\circ$) is associative:
  \[
  h \circ (g \circ f) = (h \circ g) \circ f \qquad \text{for} \qquad g
  \colon A \to B, f \colon B \to C, h \colon C \to D \in
  \arrs{\cat{C}}.
  \]
  For every object $A \in \obs{\cat{C}}$ there is an arrow $\id_A
  \colon A \to A$ such that for any arrow $f \colon A \to B$,
  \[
  \id_B \circ f = f \qquad \text{and} \qquad f \circ \id_A = f.
  \]
\end{definition}

\begin{example}
  $\Set$ is the category of sets and total functions.
  \begin{category}
    \catobj{$\obs{\Set}$ is the class of all sets,}
    \catarr{$\obs{\Set}$ is the class of total functions on sets,}
    \catcomp{$\circ$ is the (set) function composition,} \catid{is the
      family of identity (set) functions.}
  \end{category}
  Given a total function $f$, we have:
  \[ \dom(f) = \dom(f) \]
  (the domain in the ``category world'' is the same as the domain in
  the ``function world'').
\end{example}

\begin{example}
  Category $\ccat{1}$
  \begin{category}
    \catobj{$\obs{\ccat{1}} = \set{\star}$}
    \catarr{$\obs{\ccat{1}} = \set{\id_\star \colon \star \to \star}$}
    \catcomp{$\id_\star \circ \id_\star = \id_\star$}
    \catid{$\set{\id_\star}$}
  \end{category}
\end{example}

\begin{example}
  Category $\ccat{2}_{\to}$
  \begin{category}
    \catobj{$\ccat{2}_0 = \set{\star_1, \star_2}$}
    \catarr{$\ccat{2}_1 = \set{\id_1 \colon \star_1 \to \star_1, \id_2
        \colon \star_2 \to \star_2, f \colon \star_1 \to \star_2}$}
    \catcomp{follows from identity axiom}
    \catid{$\set{ \id_1, \id_2}$}
  \end{category}
\end{example}

\section{Diagrams}
We can represent properties of arrows through diagrams. For example:
\begin{diags}
  \begin{tikzcd}
    A \rar{f} \dar[swap]{g'} & B \dar{g} \\
    C \rar{f'}               & D
  \end{tikzcd}
\end{diags}
which states that $g \circ f = f' \circ g'$.

\begin{example}
  $\Grph$ is the category of multi-digraphs\footnote{Graphs with
    multiple, directed edges} and graph homomorphisms.

  \begin{category}
    \catobj{A graph $G$ is a pair of sets $G_0, G_1$ of nodes and
      arrows with a pair of maps $s_G, t_G \colon G_1 \to G_0$.}

    \catarr{A graph morphism $f \colon G \to H$ between graphs $G$ and
      $H$ is a pair
      \[
      f_0 \colon G_0 \to H_0, \quad f_1 \colon G_1 \to H_1
      \]
      of (set) functions, such that}
    \begin{diags}
      \begin{tikzcd}
        G_1 \rar{f_1} \dar[swap]{s_G} & H_1 \dar{s_H} \\
        G_0 \rar{f_0}                 & H_0
      \end{tikzcd}\hspace{2em}
      \begin{tikzcd}
        G_1 \rar{f_1} \dar[swap]{t_G} & H_1 \dar{t_H} \\
        G_0 \rar{f_0}                 & H_0
      \end{tikzcd}
    \end{diags}

    \catcomp{Given two graph morphisms
      \begin{align*}
        f \colon G \to H &= (f_0, f_1),\\
        g \colon H \to K &= (g_0, g_1),
      \end{align*}
      we define their composition as
      \[
      (g \circ f) = (g_0 \circ f_0 : G_0 \to K_0,\ g_1 \circ f_1 : G_1
      \to K_1).
      \]
    }

    \catid{The identity graph morphism $\id_G \colon G \to G$ is
      defined as the pair $(\id_{G_0}, \id_{G_1})$, where $\id_{G_0}$
      is the identity function over nodes and $\id_{G_1}$ is the
      identity function over arrows.}
  \end{category}
  We have to show that the composition is still a graph morphism and
  that it is associative.  We have
  \begin{diags}
    \begin{tikzcd}
      G_1 \rar{f_1} \dar[swap]{s_G} & H_1 \dar{s_H} \\
      G_0 \rar{f_0}                 & H_0
    \end{tikzcd}\hspace{2em}
    \begin{tikzcd}
      H_1 \rar{g_1} \dar[swap]{s_H} & K_1 \dar{s_K} \\
      H_0 \rar{g_0}                 & K_0
    \end{tikzcd}
  \end{diags}
  By pasting the two together, we can show that the condition on source
  maps holds.
  \begin{diags}
    \begin{tikzcd}
      G_1 \rar{f_1} \dar[swap]{s_G} & H_1 \dar{s_H} \rar{g_1}  & K_1 \dar{s_K} \\
      G_0 \rar{f_0}                 & H_0           \rar{g_0}  & K_0
    \end{tikzcd}
  \end{diags}
  Equivalent proof for target maps.  This proves that a composed graph
  morphism is a graph morphism.  Associativity follows from
  associativity of the underlying set functions.

  We show $\id_H \circ f = f$ using diagrams:
  \begin{diags}
    \begin{tikzcd}
      G_1 \rar{f_1} \dar[swap]{s_G} & H_1 \dar{s_H} \rar[equal]{\id_{H_1}} & H_1 \dar{s_H} \\
      G_0 \rar{f_0}                 & H_0           \rar[equal]{\id_{H_1}} & H_0
    \end{tikzcd}
  \end{diags}
\end{example}


\section{Products}

\begin{example}
  Given sets $S$, $T$ the cartesian product $S \times T$ is the set
  \[
  S \times T = \setof{\tuple{s, t}}{s \in S, t \in T}
  \]
\end{example}

We generalize this notion to products in categories.

\begin{definition}[Product]
  Given two objects $A, B$ in a category $\cat{C}$, \textbf{a} product is:
  \begin{itemize}
  \item an object $C \in \cat{C}_0$ (often denoted $A \times B$),
  \item a \emph{first projection} arrow $proj_1 \colon C \to A \in
    \cat{C}_1$ (sometimes denoted $\pi_1$),
  \item a \emph{second projection} arrow $proj_2 \colon C \to B \in
    \cat{C}_1$ (sometimes denoted $\pi_2$),
  \end{itemize}
  such that for any object $D \in \cat{C}_0$ and arrows
  \[
  q_1 \colon D \to A \in \cat{C}_1, \quad q_2 \colon D \to B \in \cat{C}_1,
  \]
  there is a \emph{unique} arrow $q \colon D \to C$ such that
  \[
  \pi_1 \circ q = p_1 \ \qquad \text{ and } \qquad \ \pi_2 \circ q = p_2.
  \]
  We can express this with the following commuting diagram:
  \begin{diags}
    \begin{tikzcd}
      {} & D \ar[swap]{dl}{q_1}\dar[dashed]{q}\ar{dr}{q_2} & \\
      A  & C \lar{\pi_1}\rar[swap]{\pi_2}                  & B
    \end{tikzcd}
  \end{diags}
  The unique arrow $q$ for given $q_1$ and $q_2$, is called the
  \emph{pairing} of $q_1$ and $q_2$ and often denoted by
  $\pair{q_1}{q_2}$.
\end{definition}

\begin{example}
  Is the cartesian product is a product in the category of sets? Yes!\\
  Given two sets $S$, $T$, the cartesian product $S \times T$ is a
  product:
  \begin{itemize}
  \item $S \times T$ is a set
  \item $\pi_1 \colon S \times T \to S, \quad \pi_1(\tuple{s, t}) = s$
  \item $\pi_2 \colon S \times T \to T, \quad \pi_1(\tuple{s, t}) = t$
  \end{itemize}
  Given a set $D$ and functions $q_1 \colon D \to S$, $q_2 \colon D
  \to T$, define $\pair{q_1}{q_2} \colon D \to S \times T$ as
  \[
  \pair{q_1}{q_2}(x) = \tuple{q_1(x), q_2(x)}.
  \]
  then
  \[
  \pi_1(\pair{q_1}{q_2}(x)) = q_1(x) \qquad \text{and} \qquad
  \pi_2(\pair{q_1}{q_2}(x)) = q_2(x).
  \]
  \begin{proof}[Proof of uniqueness of $\pair{q_1}{q_2}$]
    Let $q$ be any other function that makes the above diagram
    commute.  We need to show that for all $x \in D$,
    \[
    q(x) = \pair{q_1}{q_2}(x) = \tuple{q_1(x), q_2(x)}.
    \]
    For any $x \in D$, let $\tuple{x_1, x_2} = q(x)$.  By assumption,
    $q_i(x) = \pi_i(q(x)) = x_i$ for $i \in \set{1, 2}$, so $q(x) =
    \tuple{q_1(x), q_2(x)}$.
  \end{proof}
  An example with concrete $S$, $T$ and $D$:
  \begin{align*}
    A &= Int\\
    B &= String\\
    A \times B &= Int \times String\\
    D &= Boolean\\
    q_1 \colon Boolean \to Int &= \text{C-encoding of Boolean}\\
    q_2 \colon Boolean \to String &= x \mapsto x.\text{toString}\\
    q \colon Boolean \to Int \times String &= x \mapsto \tuple{q_1(x),
      q_2(x)}
  \end{align*}
\end{example}

\section{Isomorphisms}
\begin{definition}[Isomorphism]
  An iso in a category $\cat{C}$, is an arrow
  \[ f\colon A \to B \in \cat{C}_1 \]
  such that there is an arrow
  \[ f^{-1} \colon B \to A \]
  such that
  \[ f \circ f^{-1} = \id_B \text{ and } f^{-1} \circ f = \id_A. \]
  If there is an iso $f \colon A \to B$ we say $A$ and $B$ are isomorphic and
  we write $A \cong B$ or $f \colon A \stackrel{\sim}{\to} B$.
\end{definition}

\end{document}

%%% Local Variables:
%%% mode: latex
%%% TeX-master: t
%%% End:
